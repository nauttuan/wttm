\section{Experimental Setup}
\wchnote{Every section and subsection and subsubsection etc. gets its
own topic paragraph.}
\subsection{Hardware Specifications}
The results from our experiments should only be fully accepted with respect to
the microprocessors we specify in this section, although the conclusions will
generally apply to different generations of the hardware.

The Intel experimental machine contains a Haswell i7-4770 processor with

\begin{spacing}{0}
\begin{itemize}
\item 4 3.40GHz cores; 4 total hardware threads
\item 64 byte wide cache lines
\item 8mB shared 16-way \textit{L3} cache
\item 32kB per-core 8-way \textit{L1} cache
\end{itemize}

The IBM experimental machine contains a Power8 processor with
\begin{itemize}
\item 10 3.425GHz cores; 80 total hardware threads
\item 128 byte wide cache lines
\item 80mB shared 8-way \textit{L3} cache (roughly 8mB per-core)
\item 64kB per-core 8-way \textit{L1} cache
%http://www.7-cpu.com/cpu/Power8.html
%https://books.google.com/books?id=LIhFBAAAQBAJ&pg=PA36&lpg=PA36&dq=ibm+power8+cache+associativity&source=bl&ots=h7WlkWlmYD&sig=w0gxYk9H2-BkKEeUgnnLgS0oWC0&hl=en&sa=X&ei=ibm1VJznOYOigwSctoKoAQ&ved=0CEgQ6AEwBQ#v=onepage&q=ibm%20power8%20cache%20associativity&f=false
\end{itemize}
\end{spacing}

\subsection{Hardware Transactional Memory Interface}
All experiments are written in C and compiled with GCC, optimization level
\textit{-O0}. Our experiments use the GCC hardware transactional memory
intrinsics interface.
 
