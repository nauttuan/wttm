\section{Conclusion}
Hardware transactional memory is an exciting new solution to the synchronization
problem in multicore software. Before we can effectively use this new
technology, we must first understand the physical limits of hardware
transactions that are inherent to the Intel and IBM microprocessors on which
they are implemented. Our capacity constraint benchmarks revealed that the read
sets are implemented through the \textit{L3} and the write sets are implemented
through the \textit{L1} on the Intel machine; on the other hand, there is a
per-core, 4-set, 16-way dedicated cache that maintains the read and write sets
on the IBM machine. *INSERT SOME DISCUSSION/IMPLICATION/REVELATION, like how IBM
is not useful for sufficiently large transactions*. We anticipate that these
findings will move us in the right direction to better understanding hardware
transactional memory, ultimately enabling its proliferation into future
concurrent programs.
 
