\section{Conclusion}
Hardware transactional memory is an exciting 
new solution to the synchronization
problem in multicore software. Before we can 
effectively use this new
technology, we must first understand the physical 
limits of hardware
transactions that are inherent to the Intel 
and IBM microprocessors on which
they are implemented. Our capacity constraint 
benchmarks revealed that the read
sets are maintained through an extension of the {L3} cache and the 
write sets are maintained through a similar extension of
the {L1} cache on the Intel Haswell i7-4770.  By contrast, the
IBM Power8 appears to use a dedicated 4-set, 16-way 
cache per hardware thread to maintain the read and
write sets.

Developers using HTM on Intel's Haswell microprocessors have a lot of
flexibility with hardware transaction size, but they 
should be aware of how the
behavior of non-transactional code sharing cache with a thread
running transactional code
might affect HTM performance as well as how the access 
pattern of transactional code can limit transaction size.
IBM's Power8 developers should be cautious of the tight
restriction on transaction size, but fortunately they only 
need to reason about
HTM performance within the scope of a single hardware thread.
 
