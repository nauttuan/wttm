\section{Conclusion}
Hardware transactional memory is an exciting new solution to the synchronization
problem in multicore software. Before we can effectively use this new
technology, we must first understand the physical limits of hardware
transactions that are inherent to the Intel and IBM microprocessors on which
they are implemented. Our capacity constraint benchmarks revealed that the read
sets are implemented through the {L3} and the write sets are implemented through
the {L1} on the Intel machine; on the other hand, there is a
per-hardware-thread, 4-set, 16-way dedicated cache that maintains the read and
write sets on the IBM machine.

Developers using HTM on Intel's Haswell microprocessors have a lot of
flexibility with hardware transaction size, but they should be wary of how the
caching behavior of non-transactional code might affect HTM performance.
Practitioners on IBM's PowerPC microprocessors should be cautious of the tight
restriction on transaction size, but fortunately they only need to reason about
HTM performance within the scope of a single hardware thread.

We anticipate that these findings will move us in the right direction to better
understanding hardware transactional memory, ultimately enabling its
proliferation into future concurrent programs.
 
