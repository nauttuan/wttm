% \begin{center}
% \begin{minipage}{.9\textwidth}
% \begin{center}
% {\large \textbf{Abstract}}
% \end{center}
\begin{abstract}
Hardware transactions offer a performance advantage over software
implementations by harnessing the power of existing cache coherence mechanisms
which are already fast, automatic, and parallel. The source of superior
performance, however, is also the root of their weakness: existing
implementations of hardware transactions abort when the working set exceeds the
capacity of the underlying hardware. Before we can incorporate this nascent
technology into high-performing concurrent data structures, it is necessary to
investigate these capacity constraints in order to better inform programmers of
their abilities and limitations.

This paper provides the first empirical study of 
the ``capacity envelope'' of HTM
in Intel's Haswell and IBM's Power8 architectures, 
providing what we believe is
a much needed understanding of the extent to which 
one can use these systems to replace locks.
\end{abstract}
% \end{minipage}
% \end{center}
 
