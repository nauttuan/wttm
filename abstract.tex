% \begin{center}
% \begin{minipage}{.9\textwidth}
% \begin{center}
% {\large \textbf{Abstract}}
% \end{center}
\begin{abstract}
Hardware transactional memory is a promising solution to the synchronization
problem in multicore software due to its simple semantics and good performance
relative to traditional approaches.
Hardware transactions perform well because the overhead of implementing
transactional memory semantics is incurred at the hardware level, which is much
faster than traditional software implementations.
This reason for their good performance, however, is also the root of their
problems--hardware transactions will abort when the working set exceeds the
capacity of the underlying hardware.
Before we can incorporate this nascent technology into high-performing
concurrent programs, it is necessary to first investigate the physical capacity
constraints of hardware transactions so that we know when they are not a
feasible solution at all.
Our investigation led to revelations about where hardware transactions are
physically implemented on Intel x86 and IBM PowerPC architectures; we anticipate
these findings will increase general understanding of hardware transactional
memory, thus enabling its proliferation into future concurrent programs.
\end{abstract}
% \end{minipage}
% \end{center}
 
