% \begin{center}
% \begin{minipage}{.9\textwidth}
% \begin{center}
% {\large \textbf{Abstract}}
% \end{center}
\begin{abstract}
Hardware transactional memory (HTM) is a promising solution to the synchronization
problem in multicore software due to its simple semantics and superior performance
relative to coarse-grained locks.
Hardware transactions offer a performance advantage
over software implementations by harnessing the power
of existing cache-coherence mechanisms which are already
fast, automatic, and parallel. 
The source of superior performance, however, is also the root of their
weakness: existing implementations of hardware transactions 
will abort when the working set exceeds the
capacity of the underlying hardware.
Before we can incorporate this nascent technology into high-performing
concurrent data structures, 
it is necessary to first investigate the physical capacity
constraints of hardware transactions to inform the design
of such data structures.
We were able to divine details of the HTM implementation
of Intel's Haswell and IBM's PowerPC architectures from 
our investigation of the capacity constraints and these 
characterizations provide much needed understanding of the
systems for which practitioners design algorithms and 
data structures.
\end{abstract}
% \end{minipage}
% \end{center}
 
